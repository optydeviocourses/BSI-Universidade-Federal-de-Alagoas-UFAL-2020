%
% Documento: Estrutura
%

\chapter{CONSIDERA\c{c}\~{o}ES FINAIS}

% Sugere-se que este Capítulo 6 seja descrito em dois subtítulos ou se\c{c}\~{o}es: 6.1 - Conclus\~{o}es e  6.2 - Sugest\~{o}es para Trabalhos Futuros, que ser\~{a}o detalhados a seguir.

\section{Conclus\~{o}es}

% Nesta primeira se\c{c}\~{a}o o autor deve sintetizar as principais conclus\~{o}es do trabalho (como um todo e n\~{a}o apenas de sua aplica\c{c}\~{a}o). Deve tamb\'{e}m comentar se os objetivos geral e específicos, descritos no capítulo de introdu\c{c}\~{a}o do trabalho, foram alcan\c{c}ados, total ou parcialmente. 

% Deve comentar, ainda, os seguintes pontos: (i) se a pergunta de pesquisa formulada no Capítulo 1 foi ou n\~{a}o respondida; (ii)  avaliar se os resultados obtidos foram satisfat\'{o}rios; (iii) ressaltar os principais pontos fortes e fracos da solu\c{c}\~{a}o proposta; (iv) descrever os novos conhecimentos que foram adquiridos pela pesquisa, entre outros.

Neste trabalho aplicamos os conhecimentos obtidos com as literatura descritas no t\'{o}pico Fundamenta\c{c}\~{a}o, e focalizamos mais nos conceitos de Kimball (2013), e nos assuntos como: Extra\c{c}\~{a}o Transforma\c{c}\~{a}o e Carga (ETL), granularidade de dados, modelagem multidimensional, \textit{Data Warehouse}, reposit\'{o}rio de dados (\textit{Data Mart} e o CUBO.

Com a base formada nestas teorias, desenvolvemos um BI Básico passo-a-passo, por\'{e}m, bastante funcional aplicando os conceitos de inteligencia de neg\'{o}cio, usando ferramentas modernas de \textit{Business Intelligence}, com destaque para a Plataforma de Integra\c{c}\~{a}o de Dados e Análise Pentaho da Empresa Hitachi.

A solu\c{c}\~{a}o final pode ser usada para ajudar gestores da área de seguran\c{c}a pública, pois, trata os dados diversos advindos de um Sistema Operacional que recebe informa\c{c}\~{o}es do servi\c{c}o 181 do Estado de Alagoas. 

O BI deste trabalho processa esses dados e transforma em gráficos e pain\'{e}is funcionais, que podem ser usados para a m\'{e}trica de eventos relacionados ao uso aprimorado dos servi\c{c}o de policiamento ostensivo no âmbito da seguran\c{c}a publica estadual e municipal.

\section{Sugest\~{o}es para Trabalhos Futuros}
% Um estudo de técnicas de treinamento será interessante a trabalhos futuros.
% Outro ponto de aplicação de BI que pode ser explorado que se torna muito interessante é a área de BSC, envolvida não somente a indicadores, mas com plano de ação de acordo com a realização das metas destes indicadores.

Com este trabalho, estamos iniciando uma jornada para uma sistematiza\c{c}\~{a}o com base na inteligencia de neg\'{o}cio e que ela possa ajudar vários setores da nossa sociedade.

Nossa sugest\~{a}o para trabalhos futuros \'{e} maie técnicas de BI sejam aprimoradas e que elas inovem mais rapidamente com produto de gerenciamento de dados para análises, IA e integra\c{c}\~{a}o de dados e desenvolva sua prática de DataOps \index{DataOps}\footnote{DataOps: \'{e} uma metodologia automatizada, orientada a processos, usada por equipes analíticas e de dados, para melhorar a qualidade e reduzir o tempo de ciclo da análise de dados}. 

Com essa inova\c{c}\~{a}o sugerida esperamos que haja melhorias nas opera\c{c}\~{o}es de dados em todos os lugares Simplificando suas opera\c{c}\~{o}es de dados e forne\c{c}a acesso a informa\c{c}\~{o}es para todas as partes interessadas de uma organiza\c{c}\~{a}o com automa\c{c}\~{a}o baseada em políticas e gerenciamento de dados orientado por metadados.
