%
% Documento: Metodologia
%

\chapter{METODOLOGIA}

Para a realização deste trabalho serão aplicados os conceitos estudados sobre BI com foco em na implementa\c{c}\~{a}o de um DW/BI para o 181, usando para isso o Suite Pentaho, desenvolvida pela empresa Pentaho Corporation e mantida atualmente pela Hitachi Vantara.

No desenvolvimento desta monografia foram aplicados os conceitos estudados sobre BI, tomando como base no modelo de Kimball (2013) usando como  plataforma o BI Hitachi Pentaho, conforme tópico 4.4, suite mantida atualmente pela empresa Hitachi Vantara. 

O trabalho foi construído para que, primeiramente, seja entregue ao leitor uma introdução ao BI e suas definições assim como explicar o seu conceito e uso pr\'{a}tico através da plataforma Pentaho. Uma explicação em cada uma das principais etapas do BI, posteriormente uma apresentação sobre as ferramentas escolhidas a serem estudadas, conforme tópico 4.2, e por fim uma conclusão destes casos de uso, pr\'{a}tico atr\'{a}ves de dados reais do Disque Denúncia do Estado de Alagoas.

A metodologia é baseada na pesquisa-a\c{c}\~{a}o, junto a elementos qualitativos e quantitativos, onde ser\'{a} demonstrado os processo de extração dos dados,  cria\c{c}\~{a}o de um \textit{Data Warehouse}, com consultas e cubos OLAP an\'{a}lise usando o \textit{plugin} SAIKU, \textit{Dashboard}\index{NewR}\footnote{\textit{Dashboard} ou Painel de Controle é a apresenta\c{c}\~{a}o visual das informa\c{c}ões mais importantes e necess\'{a}rias para alcan\c{c}ar um ou mais objetivos de negócio, consolidadas e ajustadas em uma tela para f\'{a}cil acompanhamento do seu negócio}, CDE, baseado no módulo consultar denúncia, ao gerar os recursos de BI necess\'{a}rios.