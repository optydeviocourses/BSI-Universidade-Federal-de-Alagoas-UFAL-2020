%
% Documento: Metodologia
%

\chapter{METODOLOGIA}

Para a realiza\c{c}̧\~{a}o deste trabalho foram aplicados os conceitos estudados sobre BI e DW, com foco no desenvolvimento de uma Aplic\c{c}\~{a}o Web com os recursos destas tecnologias de intelig\^{e}ncia de neg\'{o}cio direcionada para o servi\c{c}o do disgue den\'{u}ncia da Seguran\c{c}a P\'{u}blica do Estado de Alagoas. 

Nesta monografia apresentamos os conhecimentos e conceitos de DW/BI desenvolvidos por \cite{dw-kimball-2013} como base para implementar a Aplicação modelo ``DW\_181''. Para o desenvolvimento da solu\c{c}\~{a}o utilizamos a plataforma de BI de c\'{o}digo aberto conhecida como \textit{Pentaho} que atualmente esta sendo comercializada como um \textit{Suite} denominado \textit{Hitachi Pentaho} e dispobibilizada em uma vers\~{a}o comunit\'{a}ria.

O trabalho foi constru\'{i}́do passo a passo, para que, primeiramente, tenhamos um aprendizado s\'{o}lida sobre Intelig\^{e} de Neg\'{o}cio e as defini\c{̧c}\~{a}o sobre esse tema ,e por fim, com a cria\c{c}\~{o}om de uma aplicação funcional que demonstre estes conceitoss sobre DW/BI.

Assim existen explica\c{c}\~{o}es em cada uma das principais etapas do desenvolvimento do BI, posteriormente uma apresenta\c{c}\~{a}o sobre as ferramentas escolhidas a serem estudadas, e por fim uma conclusão destes casos de uso, pr\'{a}tico atr\'{a}ves de dados reais do servi\c{c} 181 do Estado de Alagoas.

A metodologia \'{e} baseada na pesquisa-a\c{c}\~{a}o, junto a elementos qualitativos e quantitativos, onde ser\'{a} demonstrado os processo de extra\c{c}\~{a}o dos dados,  cria\c{c}\~{a}o de um \textit{Data Warehouse}, com consultas e cubos OLAP an\'{a}lise usando o \textit{plugin} SAIKU, \textit{Dashboard}\index{NewR}\footnote{\textit{Dashboard} ou Painel de Controle \'{e} a apresenta\c{c}\~{a}o visual das informa\c{c}ões mais importantes e necess\'{a}rias para alcan\c{c}ar um ou mais objetivos de negócio, consolidadas e ajustadas em uma tela para f\'{a}cil acompanhamento do seu negócio}, CDE, baseado no m\'{o}dulo consultar den\'{u}ncia, ao gerar os recursos de BI necess\'{a}rios.