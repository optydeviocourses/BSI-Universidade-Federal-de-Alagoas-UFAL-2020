\begin{quadro}[H]
	\begin{center}
		\caption{Comparativo entre as Características do BI Operacional, T\'{a}tico e Estratégico..\label{qua:quadro-01}}
	    \begin{tabular}{ |p{3cm}|p{3cm}|p{3cm}|p{3cm}| }
			\hline
		    Caracter\'{i}sticas & 
            BI Operacional & 
            BI T\'{a}tico & 
            BI Estrat\'{e}gico \\
		    \hline
            Foco principal do neg\'{o}cio &
            Administrar operações do dia a dia &
            Analisar dados; entregar relatórios & 
            Atingir as metas empresariais e longo prazo \\
            \hline
            Principais usu\'{a}rios &
            Gerente de setor &
            Executivos, analistas, gerentes de setor &
            Executivos, analistas \\
            \hline
            M\'{e}tricas &
            Métricas são individualizadas 
            para que o gestor de cada linha
            possa obter insight sobre o
            desempenho de seus processos de negócio
            &
            Métricas são um mecanismo de \textit{feedback}
            para companhar e entender como a estratégia
            está progredindo e quais ajustes precisam ser planejada
            &
            Métricas são um mecanismo de \textit{feedback}
            para companhar e entender como a
            estratégia está progredindo e quais ajustes
            precisam ser planejados
            \\
            \hline
            Prazo &
            Imediatamente, dentro do dia &
            Diário, semanal, mensal&
            Mensal, trimestral, anual \\
            \hline
            Tipos de dados ou usos  &
            Em tempo real ou quase em tempo real  &
            Histórico, preditivo  &
            Histórico, preditivo \\
            \hline
    	\end{tabular}
	\end{center}
	\vspace*{-0,8cm}
	{\raggedright \fonte{Elaborado à partir de\cite{bi-turban-2013}}
\end{quadro}