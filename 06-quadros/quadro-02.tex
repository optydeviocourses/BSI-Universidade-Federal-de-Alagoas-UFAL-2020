\begin{quadro}[H]
	\begin{center}
		\caption{Comparativo: BI Tradicional verso BI 2.0.\label{qua:quadro-02}}
	    \begin{tabular}{ |p{7cm}|p{7cm}| }
			\hline
		    Características Tradicional ou BI 1.0  
		    & 
            \textit{Next Generation Business Intelligence} ou BI 2.0 
            \\
		    \hline
            Envio e apresentação de relatórios estáticos para os usuários. Relatórios orientados para impressão. Análise de relatório pós-fato devido à latência dos dados. 
            &
            Comunidades de usuários dinâmicos com colaboração ativa e compartilhamento imediato de informações, onde os usuários elaboram seus próprios relatórios. Aplicações de geração de relatórios interativos e baseados na Web 2.0. Relatórios em tempo real. 
            \\
            \hline
            Custo elevado é considerado um luxo dentro da organização. (Empresas de grande porte). 
            &
            Soluções econômicas e rentáveis disponibilizadas para as organizações como um todo. (Pequenas, Médias e Grandes empresas).
            \\
            \hline
            Gráficos com barras estáticas, e gráficos circulares segmentados. 
            &
            Visualização de dados intuitiva, dinâmica e interativa.
            \\
            \hline
            Modelo OLAP para análise. 
            &
            Modelo OLAP junto com outras alternativas inovadoras, menos complexas e de alto.
            \\
            \hline
            Instalação, upgrade e uso complexo e de alto consumo de tempo. 
            &
            Instalação, upgrade e uso simplificados.
            \\
            \hline
            Parâmetros de pesquisa predefinidos. 
            &
            Pesquisas dinâmicas ou de estilo livre permitindo a exploração de dados.
            \\
            \hline
            Dados estruturados. 
            &
            Conjunto ampliado de tipos de dados suportados, inclusive dados não estruturados e serviços XML da web.
            \\
            \hline
            Licenciamento de software por usuário. 
            &
            Licenciamento de software por servidor para um número ilimitado de usuários ou licenciamento baseado em assinatura..
            \\
            \hline
            BI para todos, na medida da necessidade da organização.
            \\
           \hline
	    \end{tabular}
	\end{center}
	\vspace*{-0,8cm}
	{\raggedright \fonte{Disponível em: <https://www.devmedia.com.br>. Acesso em: 12 ago. 2020.}}
\end{quadro}
