%
% Documento: Resumo (Português)
%

\begin{RESUMO}
\thispagestyle{empty}
	\begin{SingleSpace}
	
		\hspace{-1.2 cm}Este trabalho de conclusão visa a implementação de um BI tradicional, baseado nos dados gerados pelas denúncias registradas no canal 181, disque denúncia (via: telefone, web e aplicativo), com foco na coleta, extração e organização de dados para permitir o processamento eficiente de consultas para obter percepções de dados históricos que irão auxiliar aos tomadores de decisões na elaboração de um policiamento ostensivo e preventivo, com mais eficácia e eficiência, dentro da Estrutura da Segurança Pública de Alagoas. O BI propõe um \textit{Fontend} para as informações de forma simples através da utilização da Plataforma de Integração de Dados e Análise Pentaho. O objetivo desta monografia foi alcançado, tendo as informações disponibilizadas aos gestores de uma forma gráfica, facilitando a análise dos dados. Neste trabalho usamos estratégias como a linguagem de Expressões Multidimensionais (MDX) que é uma linguagem de consulta para processamento analítico online (OLAP) usando um sistema de gerenciamento de banco de dados (SGDB), que exploram e manipulam os dados que geram informações relevantes para a organização em questão. Esses sistemas permitem a organização, análise, compartilhamento e monitoramento de informações que oferecem suporte à gestão de negócios, fazendo com que cada gestor tome decisões importantes com mais rapidez. Essa técnica de negócio descreve as habilidades das corporações para acessar os dados e explorar as informações, normalmente contidas em uma data warehouse.


		\vspace*{0.5cm}\hspace{-1.3 cm}\textbf{Palavras-chave}: \textit{Business Intelligence} (BI). \textit{Data Warehouse} (DW). MultiDimensional eXpressions (MDX). \textit{Online Analytical Processing} (OLAP).
		
		
		
	\end{SingleSpace}
\end{RESUMO}


