%
% Documento: Resumo (Inglês)
%

\begin{ABSTRACT}
	\begin{SingleSpace}
	
		\hspace{-1.3 cm}This conclusion work aims to implement a traditional BI, based on the data generated by the complaints registered in channel 181, dial the complaint (via: phone, web and application), with a focus on the collection, extraction and organization of data to allow efficient processing consultations to obtain perceptions of historical data that will assist decision makers in developing ostentatious and preventive policing, more effectively and efficiently, within the Public Security Framework of Alagoas. BI proposes a Font-end for information in a simple way through the use of the Data Integration and Analysis Pentaho Platform. The objective of this monograph was achieved, having the information made available to managers in a graphic way, facilitating data analysis. In this work we use strategies such as the Multidimensional Expressions language (MDX), which is a query language for online analytical processing (OLAP) using a database management system (SGDB), which explore and manipulate the data that generate relevant information for the organization in question. These systems allow the organization, analysis, sharing and monitoring of information that support business management, making each manager make important decisions faster. This business technique describes the ability of corporations to access data and explore information, which is normally contained in a data warehouse.


		\vspace*{0.5cm}\hspace{-1.3 cm}\textbf{Keywords}: Business Intelligence (BI). Data Warehouse (DW). MultiDimensional eXpressions (MDX). Online Analytical Processing (OLAP).
		
		
	\end{SingleSpace}

\end{ABSTRACT}
