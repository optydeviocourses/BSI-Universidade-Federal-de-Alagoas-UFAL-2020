% Documento: Agradecimento


\begin{agradecimento}

A todos os meus familiares por sempre me apoiarem, mesmo não entendendo muitas vezes o meu afastamento para estudar e pesquisar. Em especial à Minha mãe que nos deixou para ficar ao lado de Deus este ano; aos meus Filhos Caynan Eduardo e Carla Eduarda, pois, são a razão das minha lutas diárias e principalmente a Deus "Todo Poderoso" e seu Filho Jesus Cristo, por me dar: força, determinação, paciência, resiliência e paz todos os dias de minha vida.
Aos meus amigos e fãs da Plataforma de Integração de Dados e Análise Pentaho: Prof. Me. Thiago Araújo Silva de Oliveira, Mestre em Ciências da Computação pela UFPE, especialista em Pentaho, e aos Tenente PM Sidcley e ao Cabo PM Jamerson Dias Ramos, gestores de estatística do NEAC (Núcleo de Estatística e Análise Criminal) por proporcionarem ajudas técnicas especializadas em BI, para que eu concluísse esse trabalho.
Ao Sr Tenente Coronel QOC PM Marcelo da Rocha Nogueira, Assessor da AII/SSP-AL (Assessoria de Inteligência da Secretaria de Segurança Pública)  por liberar o acesso aos dados do 181, para que estes fossem trabalhados na monografia.
Ao Sr, Major QOC PM Roberto Feliciano co-fundador do 181 em nosso Estado, maior mente por trás do 18.1 atual ex-gestor do Núcleo de Disque Denúncia/All (Assessoria de Inteligência da Secretaria de Segurança Pública)  dar acesso aos dados do 181, para que os dados fossem trabalhados na monografia.
Ao meu orientador o Prof. Dr. Fábio José Coutinho da Silva, pelos incentivos, conselhos, ensinamentos e paciência durante todo período de desenvolvimento de trabalho de conclusão de curso.
A AII/SSP (Assessoria Integrada de Inteligência),  por ter disponibilizado acesso a base de dados do SISGOU.
Aos nossos professores do curso EAD-BSI-UFAL da Universidade Aberta do Brasil do polo de Maceió/AL, que contribuíram para os conhecimento adquiridos por este discente durante os períodos deste excelente curso.


\end{agradecimento}

